% Chapter 5
\newpage
\begingroup%
\makeatletter%
\let\clearpage\relax% Stop LaTeX from going to a new page; and
\vspace*{\fill}%
\vspace*{\dimexpr-50\p@-\baselineskip}% Remove the initial (default) 50pt gap (plus 1 line)
\chapterfont{\centering}
\chapter{Future Work}
\vspace*{\fill}%
\endgroup

\newpage
\label{Chapter6}
\lhead{Chapter 6. \emph{Future Work}} % Write in your own chapter title to set the page header
We aimed to develop a virtual reality-based E-Commerce web application called MetaMart, which provides customers with a realistic shopping experience in the Metaverse. This project involves developing a basic E-Commerce website with all necessary features, integrating a 3D tour and VR mode created in Unity 3D into the website, placing 3D clothing on an avatar with custom dimensions focused on providing customers with a personalized shopping experience.
The use of virtual reality technology in your application allows you to provide your customers with an immersive and realistic shopping experience. This is a vast improvement over traditional online shopping experiences based on 2D images and descriptions. By implementing the metaverse concept, applications can get a glimpse into the future of shopping in virtual worlds. 
\section{Future Work}
The current project has achieved the desired goals. However, there is still room for improvement and further development in the future. The next section outlines future work that can be done to improve the MetaMart application.

\textbf{1. Product extensions:}

The current MetaMart application focuses on providing an apparel and outerwear shopping experience. In the future, the application can be extended to other products such as accessories, shoes and electronics. This allows you to have more product diversity for your customers and attract a wider audience.

\textbf{2. Artificial intelligence integration:}

MetaMart applications can be further enhanced by integrating artificial intelligence (AI) technology to provide personalized recommendations based on the customer's shopping history, preferences, and behavior. AI algorithms can analyze customer data and suggest products that match customer interests to improve customer satisfaction and sales.

\textbf{3. Enhanced virtual reality experience:}

VR experiences can be enhanced by incorporating haptic feedback and better graphics to create a more immersive and realistic shopping experience. This can be achieved through the use of advanced VR technologies such as haptic gloves and full-body tracking systems.

\textbf{4. Integration with social media:}

MetaMart's integration with social media platforms allows customers to share their shopping experience with friends and family, creating a sense of community around the application. Additionally, you can use social media integration to promote your products and increase brand awareness.

\textbf{5. Integration with blockchain technology:}

Blockchain technology can be integrated into his MetaMart application for greater security and transparency. This can be achieved by implementing a decentralized system that tracks transaction and product authenticity, ensuring that customers receive genuine products. 6. Multilingual support:

MetaMart can be extended to support multiple languages, making it accessible to more users. This allows customers in different regions to access the application in their native language, improving user experience and customer satisfaction.

\textbf{7. Performance optimization:}

MetaMart application performance can be further optimized by reducing load times, reducing latency, and minimizing crashes and errors. This improves the overall user experience and ensures your application runs smoothly and efficiently.

In summary, the MetaMart project has demonstrated the potential of virtual reality technology in the e-commerce industry. Future work outlined above may help further develop and improve the application and provide customers with a more satisfying and immersive shopping experience in the Metaverse. 